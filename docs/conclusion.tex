\chapter{Conclusions}
\label{sec:conclusions}

In 2009, reflecting on the operating experiences of the first \gls{pbr}, the \gls{avr}, the German chemist Rainer Moormann commented that

\begin{displayquote}
Ironically, the pebble bed HTR concept has probably survived until now mainly as consequence of one of its weak points, its insufficient in-core instrumentation abilities.\hspace{0.01cm}\footnote{R. Moormann, ``AVR prototype pebble bed reactor: A safety re-evaluation of its operation and consequences for future reactors,'' \textit{Kerntechnik}, vol. 74, pp. 8--21, 2009.}
\end{displayquote}

Deficiencies in early computational models of the \gls{avr} contributed to core temperatures exceeding licensing limits by hundreds of degrees, an effect that was only quantified following a complicated melt wire pebble experiment with a data lead time of 15 months. However, to design \glspl{pbr} with an exclusive emphasis on experimentation, with all the attendant difficulties of data collection in a high temperature, high radiation, and stochastically moving environment, would be an incredibly expensive, and possibly unfruitful, endeavor. 

One might pessimistically interpret this quotation as an indication that the \gls{pbr} concept should be rejected from future study, thereby abandoning its many advantages in the areas of high temperature efficiency, process heat applications, slowly-evolving transients, passive heat removal, and low excess reactivity. Instead, Moormann's position is here viewed as an impetus to develop more accurate computational models of \glspl{pbr} that can augment complementary experimental research programs.

This dissertation developed multiscale models of \glspl{pbr} with the hope that delivering fast-running and accurate predictions of pebble bed thermal and flow physics to the nuclear industry will translate to an increased viability of the \gls{pbr} concept. With the objective of improving modeling fidelity in areas of safety relevance to \glspl{pbr}, specific emphasis was placed on 1)~model validation against experimental data, 2)~identifying knowledge gaps that affect model accuracy, and 3)~incorporating core bypass flow predictions to bound the core coolant diversion, an effect whose underestimation was likely a leading contributor to the high temperatures observed in the \gls{avr} \cite{viljoen}.

For \gls{ms} to be a valuable design tool for reactor development, numerical models must be able to characterize the proximity of the reactor state to the thermal, mechanical, and radiation limits bounding the desired operating space. Given today's computing resources and a length scale separation of approximately six orders of magnitude\mdash from a \glspl{cfp} layer thickness on the scale of 10$^{-5}$ \si{\meter} to a full core height on the scale of 10$^1$ \si{\meter}\mdash precludes routine design and analysis with fully-resolved models for all length scales and heterogeneities.

In Chapter \ref{sec:PhysicalModels}, multiscale analysis of \glspl{pbr} was introduced as a means for obtaining representative \gls{th} physics predictions over many orders of magnitude in space with full-reactor runtimes on the order of a few \gls{cpu} minutes. These models were presented in terms of three important characteristic length scales\mdash 1)~the macroscale, incorporating the core and surrounding structural materials; 2)~the mesoscale, encompassing a fuel pebble; and 3)~the microscale, spanning a \gls{cfp}. Spatial homogenization of the Navier-Stokes equations with conjugate heat transfer between the coolant and the pebbles was performed to obtain a porous media macroscale model. Two meso and micro scale models were considered. The first was a homogeneous multi-layer conduction model motivated by a simple numerical implementation in existing single-\gls{pde} software tools and a previous dissertation based upon the method \cite{xin_wang_thesis}. The second was a linear superposition approach that augments a long-wavelength background temperature solution with microscale corrections accounting for the locality of heat generation and layer thermal resistances. 

In Chapter \ref{sec:ph}, the incorporation of these multiscale models into a new software application, Pronghorn, built upon the open-source \gls{fe} \gls{moose} framework was described. The objective of this new tool development was to deliver the multiscale \gls{th} models described in Chapter \ref{sec:PhysicalModels} within a high-performance, portable, and extensible computing platform that leverages state-of-the-art numerical methods, nonlinear solvers, and meshing. Flexible in-memory multiphysics data communication systems, general 3-D unstructured meshes, and a modern software engineering design greatly expand upon the capabilities available to the nuclear community for modeling of \glspl{pbr}. Chapter \ref{sec:ph} described the spatial discretization, numerical stabilization, nonlinear solution methods, and length scale coupling used to translate the equation and closure models in Chapter \ref{sec:PhysicalModels} into predictive software.

The use of computational models to predict reactor response to nominal and off-normal conditions requires high software quality and a strong \gls{vv} base. Chapter \ref{sec:vv} discussed several verification activities used to highlight the high caliber of Pronghorn's numerical implementation, illustrate the reduction of the macroscale models to open flows, and support a common simplification used to reduce boundary layer meshing requirements. A small selection of the \gls{mms} verifications shows that the numerical implementation is convergent and matches theoretical spatial and temporal convergence rates for \gls{fe} and \gls{fd} discretizations. Excellent agreement is obtained between Pronghorn and a reference solution to a nearly-incompressible, thermally-driven natural circulation numerical benchmark; this demonstrates that the macroscale model is relevant to the open natural convection flows that may be observed in reactor plena, the accurate prediction of which has significant implications for assessing cyclic fatigue of structural materials. And in the limit of decreasing Mach number, a comparison of inviscid flow over a cylinder to an analytic potential flow solution exemplifies the general applicability of the macroscale model to both compressible and nearly-incompressible flows, which is an essential prerequisite in the development of an all-speed flow solver for reactors that exhibit the full range from low-speed natural circulation to high-speed blowdown depressurization.

Chapter \ref{sec:sana} then applied the multiscale model to the SANA experiments, a scaled facility modeling depressurized conduction cool-down of gas-cooled \glspl{pbr} as a function of power density, coolant, and pebble properties. By considering a total of 52 experiments with nearly 1300 solid temperature data points, Pronghorn's friction-dominated macroscale model is capable of predicting solid temperatures with a mean error of 22.6\si{\celsius} and a standard deviation of 54.6\si{\celsius}. A code-to-code comparison with two other commonly-used porous media applications shows a similar accuracy, but with Pronghorn's additional advantages in the areas of 3D unstructured meshing and comprehensive multiphysics coupling to other \gls{moose} applications.

By exploring the error and standard deviation as a function of space, the higher error and standard deviation in the near-wall regions of the SANA facility exhibit the need for multiscale coupling to high-resolution \gls{cfd} and experimental programs to obtain refined anisotropic drag and heat transfer closures in these regions. The primarily radial temperature gradients also suggest that more accurate predictions can be achieved with improved wall heat flux \glspl{bc}. 

Next, by individually varying macroscale closures from a baseline set and repeating all 52 steady-state and axisymmetric simulations, a deeper understanding of the influence of various macroscale closures on solid temperature predictions in gas-cooled pebble beds was obtained. The solid temperature is sensitive to the porosity and effective thermal conductivity models, with the average error increasing by 11.6\si{\celsius} and 12.9\si{\celsius}, respectively. Even though these closures only varied from the baseline set in the near-wall region, the solid temperature predictions are affected throughout the entire bed. This spatial coupling demonstrates that accurate closures in the near-wall region affect heat and mass transfer over much greater distances. Of the nine single-closure variations from the baseline set, changing the uncharacterized pebble emissivity from 0.8 to 0.9 was the only modifications that simultaneously decreased the average error (from 22.6\si{\celsius} to 9.5\si{\celsius}) and the standard deviation (from 54.6\si{\celsius} to 48.8\si{\celsius}). Equally important to \gls{th} modeling of pebble beds is complete facility characterization to reduce such systemic errors, and future experimental programs are eagerly anticipated to further the \gls{vv} of Pronghorn's multiscale models.

The multiscale model described in Chapter \ref{sec:PhysicalModels} applies to all single-phase \glspl{pbr}. In the last 20 years, considerable interest has grown in the use of molten salt coolants for \glspl{pbr} due to their high volumetric heat capacity, low vapor pressure, and high boiling points. To assess the capability of the multiscale model for capturing physics important to salt-cooled \glspl{pbr}, Chapter \ref{sec:pbfhr} applies the multiscale model to the Mark-1 \gls{pbfhr}, a small modular reactor design under development at \gls{ucb}. This particular concept is selected to 1)~highlight model strengths relative to ``legacy'' \gls{pbr} simulation tools, 2)~demonstrate the use of high-resolution \gls{cfd} for macroscale closure generation, and 3)~illustrate the capacity of the multiscale model to account for very different \gls{th} conditions than seen in most gas-cooled \glspl{pbr} such as a ``thermally-thin'' fuel-matrix region and multidimensional core and reflector flow.

In Section \ref{sec:mesomicro}, the \gls{hsd} and \gls{hl} fuel models were compared against reference, explicitly-resolved, \gls{pbfhr} pebbles. The \gls{hsd} model predicts material-wise average and maximum temperatures to within 10\si{\celsius} over a wide range in particle \gls{pf}, demonstrating both the robust nature of the model to variations in pebble design and its applicability to the nominal \gls{pbfhr} design. Conversely, the \gls{hl} method exhibits non-physical trends with \gls{pf} and generally fails to capture the coated particle layer resistance at low \gls{pf}. Predicting core bypass flows has been identified as a contributor to the high temperatures observed in the \gls{avr} because coolant is diverted from the fuel. Bypass flow is expected to play an equivalently important role in salt-cooled \glspl{pbr}, but to the author's knowledge, no studies have attempted to quantify the bypass fraction in the \gls{pbfhr}. In Section \ref{sec:bypass}, resolved turbulence modeling of the outer reflector block was performed with COMSOL \gls{cfd}. A block gap size of 5 \si{\milli\meter} was considered as representative of an end-of-life, maximum-deformation condition, while a 10 \si{\milli\meter} gap was also simulated to explore the sensitivity of the bypass to the gap width. Anisotropic friction factor models were correlated as a function of Reynolds number for axial and radial flows, providing an important closure needed for porous media modeling of the reflector region.

Section \ref{sec:inflow} then combined the \gls{hsd} verification from Section \ref{sec:mesomicro} with the friction factor models for steady-state core analysis. Five different inner reflector flow \glspl{bc} were considered in conjunction with three different reflector block gap distributions to 1)~investigate how the inlet flow \gls{bc} affects the fluid temperature distribution and bypass fraction and 2)~recommend an inlet \gls{bc} that achieves a balance between minimizing the bypass fraction, maximum and average fluid temperature on the plenum inlet, and core pressure drop. For all \glspl{bc} and gap distributions considered, a net flow of coolant from the outer reflector into the pebble bed occurs along the bottom angled surface of the outer reflector and slightly below the plenum inlet. For a fixed gap distribution, varying the inflow \gls{bc} results in roughly a 0.025 range in bypass fraction, a 15\si{\celsius} range in the average plenum inlet temperature, and a 95\si{\celsius} range in the maximum plenum inlet temperature. Generally, a higher reflector block resistance results in a lower bypass and lower core temperatures, but comparisons among multiple \glspl{bc} show that the bypass fraction must be considered in tandem with the flow \glspl{bc} as indicators of the outlet fluid temperature.

A ``bottom-peaked'' inflow \gls{bc} is recommended for further study because the longer path-length of the fluid through the bed achieves low temperature peaking on the core outlet, increasing the margin to structural material damage limits. For this particular \gls{bc}, assuming a 5 \si{\milli\meter} gap distribution is representative of an end-of-life condition, the core bypass fraction of the \gls{pbfhr} is estimated to be in the range of 13.3--15.4\%, depending on how the bypass is defined. The similarity in the bypass fraction between the uniform 5 \si{\milli\meter} gap distribution and a linearly interpolated 5 \si{\milli\meter}/10 \si{\milli\meter} piecewise sinusoidal distribution shows that the bypass is very dependent on the gap width at the axial extremes of the bed. A different block design in these regions may allow greater control over the core bypass fraction.

Section \ref{sec:depth} then concludes with research by conducting multiscale analysis of the \gls{pbfhr} for the bottom-peaked inflow \gls{bc} as a function of the reflector block gap width. Because core bypass diverts coolant from the entirety of the bed, a higher bypass raises core temperatures nearly uniformly through the bed. Fluid, pebble surface, kernel, and graphite temperatures for the three gap distributions considered differ from one another primarily in a magnitude shift. Averaged over the various gap distributions, the maximum average graphite temperature and maximum kernel temperature are approximately 25\si{\celsius} and 68\si{\celsius} higher, respectively, than the pebble surface temperature. While differences in hundreds of degrees between peak and surface fuel temperatures are common in \glspl{lwr} and some gas-cooled \glspl{pbr}, the close proximity of the \glspl{cfp} to the pebble surface results in relatively small fuel temperature gradients in the \gls{pbfhr} design. The maximum kernel temperature is approximately 93\si{\celsius} higher than the maximum fluid temperature, which retains significant margin to fuel damage limits.

The conclusions of this research are subject to many limitations in the multiscale model and completeness of the SANA facility characterization and the \gls{pbfhr} design reports. In addition to fundamental thermophysical properties for the fluid and solid phases, the multiscale model contains at least nine closures\mdash the porosity \(\epsilon\), the interphase friction factor \(W\), the interphase heat transfer coefficient \(\alpha\), the effective fluid thermal conductivity \(\kappa_f\), the effective solid thermal conductivity \(\kappa_s\), the Brinkman viscosity \(\tilde{\mu}\), and the mesocale-averaged density \(\rho_\text{meso}\), isobaric specific heat capacity \(C_{p,\text{meso}}\), and thermal conductivity \(k_\text{meso}\). The present thermal and flow predictions are limited by the accuracy of these closures, and collaboration with future experimental and numerical modeling programs is required to refine these closures in near-wall regions and for new reactor designs, such as the \gls{pbfhr}. 

A significant limitation of the porous media macroscale modeling approach exists with regards to predicting the maximum fuel temperature. Because porous media models are based on spatial averaging, all local flow and heat transfer effects are only retained in an average sense. The distribution of temperature on an individual pebble surface is unknown. Resolved \gls{cfd} simulations of pebbles show that surface temperatures can be tens to hundreds of degrees higher at stagnation points, which translates to higher fuel temperatures than a surface-uniform solid temperature can capture. While \glspl{pbr} generally have large margins to fuel damage limits, reactors with higher power densities and operating temperatures may require more refined analysis capabilities than the methods developed in this work. 

Chapters \ref{sec:vv} and \ref{sec:sana} presented a small subset of the \gls{vv} matrix used to qualify Pronghorn's models to single-phase \gls{pbr} design and safety analysis. Without similarly available salt-cooled experimental data, all predictions made for the \gls{pbfhr} Sections \ref{sec:inflow} and \ref{sec:depth} should be regarded as preliminary and subject to revision upon additional experimental validation. Many geometric simplifications were required in Chapter \ref{sec:pbfhr} due to the nature of the ongoing \gls{pbfhr} design process at \gls{ucb}. The assumed reflector block geometry neglected carbon fiber tubing and grooves to allow simpler \gls{cfd} model construction. Combined with the uniform inflow \glspl{bc} and the use of a one-equation turbulence model, the friction factor predictions made in Section \ref{sec:bypass} may differ by 100\% or more from experimentally-measured values. Since errors on the order of 100\% or more are sometimes observed, even with the use of more accurate two-equation models for similar geometries \cite{wyk}, experimental data is necessary to validate the bypass modeling in Chapter \ref{sec:pbfhr}. Future efforts to correlate convective heat transfer coefficients will also be performed to obtain more accurate temperature predictions in both the inner and outer reflector.

Finally, comprehensive modeling of \glspl{pbr} must consider the simultaneously interacting physical processes of heat and mass transfer, neutron transport, materials performance, and structural mechanics, among many others. This dissertation focused exclusively on \gls{th} modeling, which is just one component of a multiphysics reactor analysis framework needed to fully vet proposed reactor designs against a wide variety of safety, efficiency, and economic viability criteria. By developing high-quality predictive models capable of rapid \gls{th} design and analysis, this work enables an increased fidelity in design-level \gls{pbr} simulation tools by plugging into the larger community developing Generation-IV reactor models. By deepening the understanding of pebble bed thermal and flow physics, this research contributes to the viability of a promising energy technology in addressing the climate change needs of the future.
